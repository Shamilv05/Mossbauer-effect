\documentclass{article}
\usepackage{ucs} 
\usepackage{graphicx}
\usepackage[utf8x]{inputenc} % Включаем поддержку UTF8  
\usepackage[russian]{babel}
\begin{document}
\section{Введение}
\hspace{12pt} Эффект Мёссбауэра, или ядерный гамма-резонанс, заключается в резонансном поглощении ядром гамма-кванта, который был испущен таким же ядром при переходе из возбужденного состояния в основное.
\\
\indent По-видимому, это один из самых <<острых>> физических резонансов из наблюдаемых экспериментально. Действительно, добротность мёссбауэровского резонанса можно оценить через отношение естественной ширины $\Gamma$ 
гамма-перехода ядра к энергии E этого перехода. Для наиболее употребительного в мёссбауэровской спектроскопии гамма-перехода ядра $^{57}Fe$ из возбужденного состояния с энергией E = 14,413 кэВ (период полураспада - 98,1 нс, $\Gamma$ $\approx$ $7\cdot 10^{-9}$ эВ) в основном добротность резонанса достигает величины $1,5\cdot 10^{12}$. В исключительных же случаях добротность мёссбауэрского резонанса оценивается величиной $10^{15}$ для $^{67}Zn$ и даже  $2\cdot 10^{22}$ для $^{107}Ag$. Именно в силу своей высокой добротности мёссбауэровский ядерный гамма-резонанс оказался мощным, а иногда и единственным методом измерения сверхмалых сдвигов энергии ядерных гамма-переходов.
\\
\indent Эффект Мёссбауэра лежит в основе мёссбауэровской спектроскопии, которая является неразрушающим методом исследования химического состава, кристаллической структуры и магнитных свойств конденсированного состояния вещества. Другое название этого метода исследования вещества - ядерная гамма-резонансная спектроскопия, или ЯГР-спектроскопия. Разумеется, с помощью мёссбауэровской спектроскопии можно исследовать только вещества, которые содержат ядра, демонстрирующие эффект Мёссбауэра, а этот эффект наблюдаем далеко не для всех ядер. Кроме того, это физическое явление может быть полезно для прецизионных измерений гамма квантов, например, с его помощью было измерено гравитационное смещение энергии фотонов.
\section{Физический смысл эффекта Мёссбауэра}
\hspace{12pt} Рассмотрим ядро $^{191}Ir$, находящееся в возбужденном состоянии с энергий E = 129 кэВ, из которого оно может перейти в основное состояние в результате испускания $\gamma$-кванта с периодом полураспада $T_{1/2} \approx 10^{-10}$сек. Тогда согласно соотношению неопределенностей энергия возбужденного состояния E известна с погрешностью $${\Delta}E \approx {\hbar}/{\Delta}t = \frac{10^{-27}}{10^{-10}\cdot 1,6 \cdot 10^{-12}} \approx 5 \cdot 10^{-6} \hspace{2pt}eV$$
\hspace{12pt}Чем быстрее происходит высвечивание возбужденного состояния, тем больше неопределенность в значении энергии возбужденного состояния. Только основное состояние стабильного ядра имеет $\Delta$E = 0 и, следовательно, характеризуется строго определенным значением энергии.
\\
\indent Неопределенность в энергии возбужденного состояния приводит к немонохроматичности $\gamma$-излучения, испускаемого при переходе ядра из возбужденного состояния в основное. Эту немонохроматичность принято называть естественной шириной $\Gamma$ линии испускания $\gamma$-излучения. В нашем примере $\Gamma \approx 5\cdot 10^{-6}$ эВ. Это очень малая величина по сравнению с энергией $\gamma$-перехода E = 129 кэВ. Поэтому если бы существовал способ обнаружения изменения энергии на величину порядка естественной ширины линии излучения, то он дал бы возможность измерять энергию с очень высокой относительной точностью, равной $\Gamma/E$. В нашем примере $\Gamma/E = 4\cdot 10^{-11}$. Для более узких линий, т.е для $\gamma$-переходов с большими периодами, значение $\Gamma/E$ должно быть еще меньше.
\\
\indent В принципе обнаружить изменение энергии, равное естественной ширине линии излучения, можно при помощи резонансного поглощения $\gamma$-излучения. Резонансным поглощением $\gamma$-излучения называется процесс возбуждения ядра под действием $\gamma$-квантов, испускаемых этими ядрами при обратных переходах и данного возбужденного состояния в основное.
\\
\indent Процесс резонансного поглощения можно сравнительно легко наблюдать экспериментально, изучая прохождение резонансного $\gamma$-излучения через пластинку из данного вещества. При совпадении энергии $\gamma$-излучения с энергией перехода поглощение резко возрастает, что позволяет заметить очень небольшие изменения энергии вблизи резонансного значения. Однако до 1958 г. этот метод можно было использовать только при достаточно больших ширинах линий.
\\
\indent Дело в том, что при переходе ядра из возбужденного состояния с энергией E в основное состояние испускающийся $\gamma$-квант уносит не всю энергию возбуждения E, а несколько меньшую величину $E_{\gamma_{em}}$, так как часть энергии $T_{nuc}$ идет на отдачу испускающего ядра: 
$$ E_{\gamma_{em}} = E - T_{nuc} < E$$
(сравните с аналогичным явлением в $\alpha$- и $\beta$-распаде).Аналогично для возбуждения ядра до энергии E необходимо $\gamma$-излучение с энергией 
$$ E_{\gamma_{abs}} = E + T_{nuc} > E,$$
где $T_{nuc}$ - энергия отдачи, передаваемая $\gamma$-квантом поглощающему ядру. Таким образом, линия испускания и линия поглощения для одного и того же состояния в данном ядре сдвинуты относительно друг друга на 2$T_{nuc}$

\begin{figure}[h]
\center{\includegraphics[scale=0.05]{a.jpg}}
\caption{сдвиг}
\label{fig:image}
\end{figure}

\indent Энергию отдачи легко подсчитать, если учесть, что в процессе испускания $\gamma$-кванта должен выполняться закон сохранения импульса $p_{\gamma}$ = $p_{nuc}$:
$$ T_{nuc} = P_{nuc}^{2}/2M_{nuc} = p_{\gamma}^{2}/2M_{nuc} = E_{\gamma}^{2}/2M_{nuc}c^{2} \approx E^{2}/2M_{nuc}c^{2}.$$
\indent Для нашего примера получается очень небольшое значение
$$ T_{nuc} = (1,29\cdot 10^{5})^2/2\cdot 191 \cdot 931 \cdot 10^{6} \approx 0,05 \hspace{2pt}eV$$
однако оно существенно превышает естественную ширину линии излучения $\Gamma$:
$$ T_{nuc} \gg \Gamma $$
\indent Казалось бы отсюда следует абсолютная невозможность резонансного процесса. Однако это неверно потому, что реальная ширина линии испускания(и линии поглощения) определяется не естественной шириной $\Gamma$, а доплеровским уширением:
$$ D = 2\sqrt{T_{nuc}kT^0}$$,
которое при комнатной температуре (T = 300 K, kT = 0,025 эВ) равно
$$ D(300 \hspace{2pt}K) = 2\sqrt{0,05\cdot 0,025} \approx 0,07 \hspace{2pt} eV$$
\indent В связи с тем что $D \approx T_{nuc}$, доплеровски уширенные линии испускания и поглощения частично перекрываются (заштрихованная область на рисунке)  резонансный процесс становится возможен. Правда, он не обладает большой остротой, так как
$$ E/\Gamma = 0,07/1,3 \cdot 10^{5} \approx 0,5 \cdot 10^{-6},$$
и наблюдается только для очень малого количества $\gamma$-квантов, соответствующих небольшой области перекрытия линий.

\section{Два опыта Мёссбауэра}
\hspace{12pt} В 1958 г. немецкий физик Р. Мёссбауэр, проводя опыты по изучению резонансного поглощения в условиях частичного перекрытия линий в результате их доплеровского уширения, решил уменьшить D при помощи охлаждения источника и поглотителя. При этом естественно было ожидать уменьшения доли поглощенных квантов (из-за сокращения области перекрытия линий). Вместо этого в опыте было обнаружено увеличение эффекта, которое свидетельствовало о возрастании области перекрытия.
\indent Для объяснения этого странного результата Мёссбауэр предположил, что при определенных условиях (достаточно малая энергия перехода и низкая температура по сравнению с дебаевской температурой кристалла) импульс и энергия отдачи, возникающие при испускании (поглощении) $\gamma$-кванта, не идут ни на выбивание атома из узла решетки, ни на изменение энергетического состояния кристалла, а передаются упругим образом всему кристаллу в целой (точнее, очень большой группе атомов - $N \approx 10^{8}$, охватываемой бегущей звуковой волной за время испускания). В этом случае корреляция между импульсом и энергией ядра-излучателя (поглотителя) разрывается, так как из-за большой массы кристалла энергия отдачи R практически равна нулю:
$$ R = P_{nuc}^{2}/2 \cdot 10^{8} \cdot M_{nuc} = T_{nuc}/10^{8} \approx 5 \cdot 10^{-10} \hspace{2pt} eV \ll \Gamma $$
\indent В результате становятся возможными акты испускания (поглощения) $\gamma$-квантов без отдачи, т.е. сдвиг между линией испускания и линией поглощения исчезнет,
$$ E_{\gamma_{em}} = E_{\gamma_{abs}} .$$
\indent Одновременно для этих актов испускания и поглощения должно исчезнуть и доплеровское уширения D, которое теперь будет меньше естественной ширины линии $\Gamma$,
$$ D(88K) = 2\sqrt{RkT_{88^{\textdegree}}} = 2\sqrt{5 \cdot 10^{-10} \cdot 0,0075} \approx 4 \cdot 10^{-6} \hspace{2pt}eV < \Gamma$$
\indent Другими словами, при перечисленных выше условиях должен наблюдаться острый резонанс без отдачи с шириной, равной естественной ширине линии $\Gamma$. Это объяснение Мёссбауэр блестяще доказал в своем знаменитом втором опыте.
\\
\indent Схема опыта Мёссбауэра изображена на рисунке. Здесь И - источник $\gamma$-излучения $^{191}Ir$ с энергией 129 кэВ, П - иридиевый поглотитель, Д - детектор. Источник и поглотитель были помещены в криостаты $K_1$ и $K_2$, в которых поддерживалась температура T = 88 K. Криостат $K_2$ с источником мог вращаться. При вращении его в одну сторону источник приближался в поглотителю с некоторой скоростью $\vartheta$, а при вращении в другую сторону удалялся от него с той же скоростью.

\begin{figure}[h]
\center{\includegraphics[scale=0.05]{b.jpg}}
\caption{Схема опыта Мёссбауэра}
\label{fig:image}
\end{figure}

\indent В опыте измерялось поглощение $\gamma$-квантов при различных скоростях источника. Результаты опыта приведены на рисунке.
\\
\indent Здесь по оси абсцисс отложена относительная скорость источника и поглотителя и соответствующие ей изменение энергии $\Delta E$ испускаемых $\gamma$-квантов (из-за эффекта Доплера). По оси ординат отложена относительная разность интенсивности $\gamma$-излучения, проходящего через иридиевый и платиновый (для оценки фона) поглотители одинаковой толщины. Из рисунка видно, что резонанс нарушается уже при скоростях в несколько сантиметров в секунду, которые соответствуют доплеровскому изменению энергии $\gamma$-квантов, меньшему $10^{-5} \hspace{2pt}eV$. Отсюда следует, что в опыте действительно наблюдалась линия без отдачи с естественной шириной ${\gamma}$-перехода, равной $\Gamma \approx 5\cdot 10^{-6} \hspace{2pt} eV$

\begin{figure}[h]
\center{\includegraphics[scale=0.05]{c.jpg}}
\caption{Результаты опыта Мёссбауэра}
\label{fig:image}
\end{figure}

\indent Метод резонансного поглощения позволяет измерять очень малые изменения энергии. Выше указывалось, что мерой точности этого метода может служить величина $\Gamma$/E, которая для рассмотренного примера равна $4\cdot10^{-11}$. На самом деле относительная точность измерения энергии еще выше, так как экспериментально можно заметить изменение поглощение при сдвиге линии на 1/100 долю от ее естественной ширины.
\\
\indent Эффект Мёссбауэра наблюдался для многих веществ, причем для некоторых из них были зарегестрированы еще более узкие линии. Величина эффекта бывает около 1$\%$, а иногда и еще меньше, рабочие температуры колеблются для разных веществ от комнатной до гелиевой (примерно 4 К). С ростом температуры эффект постепенно ослабевает и пропадает.
\\
\indent За открытие излучения, рассеяния и поглощения без отдачи Р. Мёссбауэру была присуждена Нобелевская премия по физике за 1961 г.

\end{document}
