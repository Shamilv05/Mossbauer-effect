\documentclass{article}
\usepackage{ucs} 
\usepackage[utf8x]{inputenc} % Включаем поддержку UTF8  
\usepackage[russian]{babel}
\begin{document}
\section{Введение}
\hspace{12pt} Эффект Мёссбауэра, или ядерный гамма-резонанс, заключается в резонансном поглощении ядром гамма-кванта, который был испущен таким же ядром при переходе из возбужденного состояния в основное.
\\
\indent По-видимому, это один из самых <<острых>> физических резонансов из наблюдаемых экспериментально. Действительно, добротность мёссбауэровского резонанса можно оценить через отношение естественной ширины $\Gamma$ 
гамма-перехода ядра к энергии E этого перехода. Для наиболее употребительного в мёссбауэровской спектроскопии гамма-перехода ядра $^{57}Fe$ из возбужденного состояния с энергией E = 14,413 кэВ (период полураспада - 98,1 нс, $\Gamma$ $\approx$ $7\cdot 10^{-9}$ эВ) в основном добротность резонанса достигает величины $1,5\cdot 10^{12}$. В исключительных же случаях добротность мёссбауэрского резонанса оценивается величиной $10^{15}$ для $^{67}Zn$ и даже  $2\cdot 10^{22}$ для $^{107}Ag$. Именно в силу своей высокой добротности мёссбауэровский ядерный гамма-резонанс оказался мощным, а иногда и единственным методом измерения сверхмалых сдвигов энергии ядерных гамма-переходов.

\section{Физический смысл эффекта Мёссбауэра}
\hspace{12pt} Рассмотрим ядро $^{191}Ir$, находящееся в возбужденном состоянии с энергий E = 129 кэВ, из которого оно может перейти в основное состояние в результате испускания $\gamma$-кванта с периодом полураспада $T_{1/2} \approx 10^{-10}$сек. Тогда согласно соотношению неопределенностей энергия возбужденного состояния E известна с погрешностью $${\Delta}E \approx {\hbar}/{\Delta}t = \frac{10^{-27}}{10^{-10}\cdot 1,6 \cdot 10^{-12}} \approx 5 \cdot 10^{-6} \hspace{2pt}eV$$
\hspace{12pt}Чем быстрее происходит высвечивание возбужденного состояния, тем больше неопределенность в значении энергии возбужденного состояния. Только основное состояние стабильного ядра имеет $\Delta$E = 0 и, следовательно, характеризуется строго определенным значением энергии.
\end{document}
